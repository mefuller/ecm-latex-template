%%%%%%%%%%%%%%%%%%%%%%%%%%%%%%%%%%%%%%%%%%%%%%%%%%%%%%%%%%%%%%%%%%%%%%%
%                          template.tex
%
% LaTeX template for papers for the European Combustion Meetings
%
% Authors:
%     Mark E. Fuller, RWTH Aachen University
%
% Developed by modifying template.tex from the ussci class:
%
	% LaTeX template for papers conforming to the United States Sections of
	% the Combustion Institute style guide.
	%
	% Authors:
	%     Bryan W. Weber, University of Connecticut
	%     Kyle E. Niemeyer, Oregon State University
	%
	% The source for this template can be found at
	% https://github.com/pr-omethe-us/ussci-latex-template
%
% This work is licensed under the Creative Commons Attribution 4.0
% International License. To view a copy of this license, visit
% http://creativecommons.org/licenses/by/4.0/.
%
%%%%%%%%%%%%%%%%%%%%%%%%%%%%%%%%%%%%%%%%%%%%%%%%%%%%%%%%%%%%%%%%%%%%%%%
\documentclass[10pt,a4paper]{ecm} %twocolumn,

%======================================================================
%place additional packages here

%======================================================================
% BibLaTeX and biber (not BibTeX) are used to process the references,
% so these packages must be installed on your system. All configuration
% for the bibliography and citations are done in the ussci.cls file
% Add your bibliography file here, replacing template.bib
\addbibresource{template.bib}
%======================================================================

\title{Title of the Contributed Paper}

\author[1]{A. Author}
\author[2]{\underline{B. Author}} %underline presenting author
\author[3]{C. Author$^{*,}$} %star corresponding author; add comma if affiliation superscripts also present

\affil[1]{A. Author Affiliation}
\affil[2]{B. Author Affiliation}
\affil[3]{C. Author Affiliation}
\newcommand{\corremail}{author@university.edu} %corresponding author email address

\begin{document}

\maketitle

\thispagestyle{empty} %necessary to eliminate "1" on lead page

%====================================================================
\section*{Abstract % not to exceed 200 words
	\blfootnote{ % slightly hacky insertion of corresponding author footnote
		*Corresponding author: \email{\corremail}\\Proceedings of the European Combustion Meeting 2021
	}
}
\noindent The paper is prepared for inclusion in the proceedings of the ECM 2021 and follows the format established in previous European Combustion Meetings.
The purpose of this document is to provide rules for obtaining high-quality proceedings.
We urge you to submit your papers in the described format.
The paper should begin with a four-to-six line one-column abstract as illustrated here.
The rest of the material should be in two-column format with margin and spacing as illustrated below.

% \begin{multicols*}{2} % use this environment for unbalanced columns
\begin{multicols}{2} % use this environment for balanced columns
\section*{Introduction}
The purpose of this paper is to describe the official format of the papers to be included in the electronic Proceedings of the European Combustion Meeting 2021.
The paper format is almost identical to the one used in Combustion and Flame.
Papers are limited to six pages and 2 MB (in Portable Document Format - PDF)

\section*{Font, Layout, and Margins}
The six-page paper should be set in Times New Roman font.
The paper title should be at the top of the page, centered, in Times New Roman 12pt bold typeface with capitalized initials (Style “Title”).
Names of authors (Style “Authors”) should be typed in Times New Roman 11pt and centered under the title.
The affiliation of the authors has to be typed in Times New Roman 10pt (Style “Affiliation”) centered under the list of the authors’ names.
Use the apexes 1, 2 etc to distinguish between the affiliations of authors.
There should be a 12pt blank line between the title and the author’s name(s).
The corresponding author should be identified by a superscript [*] and a footnote at the bottom of the page, as shown below. The footnote should run across both columns and should be separated from the text by a solid line starting at the left and of length equal to 5cm.
The phrase “Proceedings of the European Combustion Meeting 2021” should be added to the footnote.
A short abstract (100 – 125 words) should follow the authors’ affiliations.
There should be a 12pt blank line between the authors’ affiliation and the abstract title.
The abstract should be typed in Types New Roman 10pt justified on both sides.
The abstract should state the most important specific objectives, the major results, and the most significant conclusions of the paper.
The remaining paper should be set in a two-column format with a Times New Roman 10pt font.
There should be a 2.5cm margin on the top, left, and bottom sides of the pages and a 2cm margin on the right side of the pages.
This will allow the two columns to be 8cm wide with a 0.5cm space between them.
Illustrations that cannot be accommodated within one column width and associated captions may extend across the page.
The orientation of the entire paper including all illustrations should be in the portrait format.
Headings of sections within the paper should be bold faced and aligned to the left of the column.
The text should begin on the next line after the heading.
here should be a 10pt blank line between the end of one section and the heading of the next section.
Paragraphs should begin with 0.5cm indention and should be justified on both sides of the column.

\section*{Page Numbers}
Page numbers should be printed, centered at the bottom of pages 2 – 6.
Do not print a page number in the first page.


\subsection*{Contents}
Each paper is expected to contain sections entitled Introduction, Conclusions, Acknowledgements, and References.
The additional section headings depend on the individual paper.
References should be prepared in the format used in {\it Combustion and Flame}.

\section*{Illustrations}
The illustrations and their captions should be included within the manuscript at a location closest to the first reference describing them.
The labels, symbols and other information on the illustrations should be no smaller than 10 points.
Illustrations may be in color or black and white, but all non-illustration text (including captions) must be in black.

\section*{Introduction}
Mandatory section

\section*{Conclusion}
Mandatory section

\section*{Acknowledgements}
Mandatory section.
This research was funded by \ldots

\printbibliography

% \end{multicols*} % use this environment for unbalanced columns
\end{multicols} % use this environment for balanced columns

\end{document}
